
% Default to the notebook output style

    


% Inherit from the specified cell style.




    
\documentclass[11pt]{article}

    
    
    \usepackage[T1]{fontenc}
    % Nicer default font (+ math font) than Computer Modern for most use cases
    \usepackage{mathpazo}

    % Basic figure setup, for now with no caption control since it's done
    % automatically by Pandoc (which extracts ![](path) syntax from Markdown).
    \usepackage{graphicx}
    % We will generate all images so they have a width \maxwidth. This means
    % that they will get their normal width if they fit onto the page, but
    % are scaled down if they would overflow the margins.
    \makeatletter
    \def\maxwidth{\ifdim\Gin@nat@width>\linewidth\linewidth
    \else\Gin@nat@width\fi}
    \makeatother
    \let\Oldincludegraphics\includegraphics
    % Set max figure width to be 80% of text width, for now hardcoded.
    \renewcommand{\includegraphics}[1]{\Oldincludegraphics[width=.8\maxwidth]{#1}}
    % Ensure that by default, figures have no caption (until we provide a
    % proper Figure object with a Caption API and a way to capture that
    % in the conversion process - todo).
    \usepackage{caption}
    \DeclareCaptionLabelFormat{nolabel}{}
    \captionsetup{labelformat=nolabel}

    \usepackage{adjustbox} % Used to constrain images to a maximum size 
    \usepackage{xcolor} % Allow colors to be defined
    \usepackage{enumerate} % Needed for markdown enumerations to work
    \usepackage{geometry} % Used to adjust the document margins
    \usepackage{amsmath} % Equations
    \usepackage{amssymb} % Equations
    \usepackage{textcomp} % defines textquotesingle
    % Hack from http://tex.stackexchange.com/a/47451/13684:
    \AtBeginDocument{%
        \def\PYZsq{\textquotesingle}% Upright quotes in Pygmentized code
    }
    \usepackage{upquote} % Upright quotes for verbatim code
    \usepackage{eurosym} % defines \euro
    \usepackage[mathletters]{ucs} % Extended unicode (utf-8) support
    \usepackage[utf8x]{inputenc} % Allow utf-8 characters in the tex document
    \usepackage{fancyvrb} % verbatim replacement that allows latex
    \usepackage{grffile} % extends the file name processing of package graphics 
                         % to support a larger range 
    % The hyperref package gives us a pdf with properly built
    % internal navigation ('pdf bookmarks' for the table of contents,
    % internal cross-reference links, web links for URLs, etc.)
    \usepackage{hyperref}
    \usepackage{longtable} % longtable support required by pandoc >1.10
    \usepackage{booktabs}  % table support for pandoc > 1.12.2
    \usepackage[inline]{enumitem} % IRkernel/repr support (it uses the enumerate* environment)
    \usepackage[normalem]{ulem} % ulem is needed to support strikethroughs (\sout)
                                % normalem makes italics be italics, not underlines
    

    
    
    % Colors for the hyperref package
    \definecolor{urlcolor}{rgb}{0,.145,.698}
    \definecolor{linkcolor}{rgb}{.71,0.21,0.01}
    \definecolor{citecolor}{rgb}{.12,.54,.11}

    % ANSI colors
    \definecolor{ansi-black}{HTML}{3E424D}
    \definecolor{ansi-black-intense}{HTML}{282C36}
    \definecolor{ansi-red}{HTML}{E75C58}
    \definecolor{ansi-red-intense}{HTML}{B22B31}
    \definecolor{ansi-green}{HTML}{00A250}
    \definecolor{ansi-green-intense}{HTML}{007427}
    \definecolor{ansi-yellow}{HTML}{DDB62B}
    \definecolor{ansi-yellow-intense}{HTML}{B27D12}
    \definecolor{ansi-blue}{HTML}{208FFB}
    \definecolor{ansi-blue-intense}{HTML}{0065CA}
    \definecolor{ansi-magenta}{HTML}{D160C4}
    \definecolor{ansi-magenta-intense}{HTML}{A03196}
    \definecolor{ansi-cyan}{HTML}{60C6C8}
    \definecolor{ansi-cyan-intense}{HTML}{258F8F}
    \definecolor{ansi-white}{HTML}{C5C1B4}
    \definecolor{ansi-white-intense}{HTML}{A1A6B2}

    % commands and environments needed by pandoc snippets
    % extracted from the output of `pandoc -s`
    \providecommand{\tightlist}{%
      \setlength{\itemsep}{0pt}\setlength{\parskip}{0pt}}
    \DefineVerbatimEnvironment{Highlighting}{Verbatim}{commandchars=\\\{\}}
    % Add ',fontsize=\small' for more characters per line
    \newenvironment{Shaded}{}{}
    \newcommand{\KeywordTok}[1]{\textcolor[rgb]{0.00,0.44,0.13}{\textbf{{#1}}}}
    \newcommand{\DataTypeTok}[1]{\textcolor[rgb]{0.56,0.13,0.00}{{#1}}}
    \newcommand{\DecValTok}[1]{\textcolor[rgb]{0.25,0.63,0.44}{{#1}}}
    \newcommand{\BaseNTok}[1]{\textcolor[rgb]{0.25,0.63,0.44}{{#1}}}
    \newcommand{\FloatTok}[1]{\textcolor[rgb]{0.25,0.63,0.44}{{#1}}}
    \newcommand{\CharTok}[1]{\textcolor[rgb]{0.25,0.44,0.63}{{#1}}}
    \newcommand{\StringTok}[1]{\textcolor[rgb]{0.25,0.44,0.63}{{#1}}}
    \newcommand{\CommentTok}[1]{\textcolor[rgb]{0.38,0.63,0.69}{\textit{{#1}}}}
    \newcommand{\OtherTok}[1]{\textcolor[rgb]{0.00,0.44,0.13}{{#1}}}
    \newcommand{\AlertTok}[1]{\textcolor[rgb]{1.00,0.00,0.00}{\textbf{{#1}}}}
    \newcommand{\FunctionTok}[1]{\textcolor[rgb]{0.02,0.16,0.49}{{#1}}}
    \newcommand{\RegionMarkerTok}[1]{{#1}}
    \newcommand{\ErrorTok}[1]{\textcolor[rgb]{1.00,0.00,0.00}{\textbf{{#1}}}}
    \newcommand{\NormalTok}[1]{{#1}}
    
    % Additional commands for more recent versions of Pandoc
    \newcommand{\ConstantTok}[1]{\textcolor[rgb]{0.53,0.00,0.00}{{#1}}}
    \newcommand{\SpecialCharTok}[1]{\textcolor[rgb]{0.25,0.44,0.63}{{#1}}}
    \newcommand{\VerbatimStringTok}[1]{\textcolor[rgb]{0.25,0.44,0.63}{{#1}}}
    \newcommand{\SpecialStringTok}[1]{\textcolor[rgb]{0.73,0.40,0.53}{{#1}}}
    \newcommand{\ImportTok}[1]{{#1}}
    \newcommand{\DocumentationTok}[1]{\textcolor[rgb]{0.73,0.13,0.13}{\textit{{#1}}}}
    \newcommand{\AnnotationTok}[1]{\textcolor[rgb]{0.38,0.63,0.69}{\textbf{\textit{{#1}}}}}
    \newcommand{\CommentVarTok}[1]{\textcolor[rgb]{0.38,0.63,0.69}{\textbf{\textit{{#1}}}}}
    \newcommand{\VariableTok}[1]{\textcolor[rgb]{0.10,0.09,0.49}{{#1}}}
    \newcommand{\ControlFlowTok}[1]{\textcolor[rgb]{0.00,0.44,0.13}{\textbf{{#1}}}}
    \newcommand{\OperatorTok}[1]{\textcolor[rgb]{0.40,0.40,0.40}{{#1}}}
    \newcommand{\BuiltInTok}[1]{{#1}}
    \newcommand{\ExtensionTok}[1]{{#1}}
    \newcommand{\PreprocessorTok}[1]{\textcolor[rgb]{0.74,0.48,0.00}{{#1}}}
    \newcommand{\AttributeTok}[1]{\textcolor[rgb]{0.49,0.56,0.16}{{#1}}}
    \newcommand{\InformationTok}[1]{\textcolor[rgb]{0.38,0.63,0.69}{\textbf{\textit{{#1}}}}}
    \newcommand{\WarningTok}[1]{\textcolor[rgb]{0.38,0.63,0.69}{\textbf{\textit{{#1}}}}}
    
    
    % Define a nice break command that doesn't care if a line doesn't already
    % exist.
    \def\br{\hspace*{\fill} \\* }
    % Math Jax compatability definitions
    \def\gt{>}
    \def\lt{<}
    % Document parameters
    \title{serie\_05}
    
    
    

    % Pygments definitions
    
\makeatletter
\def\PY@reset{\let\PY@it=\relax \let\PY@bf=\relax%
    \let\PY@ul=\relax \let\PY@tc=\relax%
    \let\PY@bc=\relax \let\PY@ff=\relax}
\def\PY@tok#1{\csname PY@tok@#1\endcsname}
\def\PY@toks#1+{\ifx\relax#1\empty\else%
    \PY@tok{#1}\expandafter\PY@toks\fi}
\def\PY@do#1{\PY@bc{\PY@tc{\PY@ul{%
    \PY@it{\PY@bf{\PY@ff{#1}}}}}}}
\def\PY#1#2{\PY@reset\PY@toks#1+\relax+\PY@do{#2}}

\expandafter\def\csname PY@tok@w\endcsname{\def\PY@tc##1{\textcolor[rgb]{0.73,0.73,0.73}{##1}}}
\expandafter\def\csname PY@tok@c\endcsname{\let\PY@it=\textit\def\PY@tc##1{\textcolor[rgb]{0.25,0.50,0.50}{##1}}}
\expandafter\def\csname PY@tok@cp\endcsname{\def\PY@tc##1{\textcolor[rgb]{0.74,0.48,0.00}{##1}}}
\expandafter\def\csname PY@tok@k\endcsname{\let\PY@bf=\textbf\def\PY@tc##1{\textcolor[rgb]{0.00,0.50,0.00}{##1}}}
\expandafter\def\csname PY@tok@kp\endcsname{\def\PY@tc##1{\textcolor[rgb]{0.00,0.50,0.00}{##1}}}
\expandafter\def\csname PY@tok@kt\endcsname{\def\PY@tc##1{\textcolor[rgb]{0.69,0.00,0.25}{##1}}}
\expandafter\def\csname PY@tok@o\endcsname{\def\PY@tc##1{\textcolor[rgb]{0.40,0.40,0.40}{##1}}}
\expandafter\def\csname PY@tok@ow\endcsname{\let\PY@bf=\textbf\def\PY@tc##1{\textcolor[rgb]{0.67,0.13,1.00}{##1}}}
\expandafter\def\csname PY@tok@nb\endcsname{\def\PY@tc##1{\textcolor[rgb]{0.00,0.50,0.00}{##1}}}
\expandafter\def\csname PY@tok@nf\endcsname{\def\PY@tc##1{\textcolor[rgb]{0.00,0.00,1.00}{##1}}}
\expandafter\def\csname PY@tok@nc\endcsname{\let\PY@bf=\textbf\def\PY@tc##1{\textcolor[rgb]{0.00,0.00,1.00}{##1}}}
\expandafter\def\csname PY@tok@nn\endcsname{\let\PY@bf=\textbf\def\PY@tc##1{\textcolor[rgb]{0.00,0.00,1.00}{##1}}}
\expandafter\def\csname PY@tok@ne\endcsname{\let\PY@bf=\textbf\def\PY@tc##1{\textcolor[rgb]{0.82,0.25,0.23}{##1}}}
\expandafter\def\csname PY@tok@nv\endcsname{\def\PY@tc##1{\textcolor[rgb]{0.10,0.09,0.49}{##1}}}
\expandafter\def\csname PY@tok@no\endcsname{\def\PY@tc##1{\textcolor[rgb]{0.53,0.00,0.00}{##1}}}
\expandafter\def\csname PY@tok@nl\endcsname{\def\PY@tc##1{\textcolor[rgb]{0.63,0.63,0.00}{##1}}}
\expandafter\def\csname PY@tok@ni\endcsname{\let\PY@bf=\textbf\def\PY@tc##1{\textcolor[rgb]{0.60,0.60,0.60}{##1}}}
\expandafter\def\csname PY@tok@na\endcsname{\def\PY@tc##1{\textcolor[rgb]{0.49,0.56,0.16}{##1}}}
\expandafter\def\csname PY@tok@nt\endcsname{\let\PY@bf=\textbf\def\PY@tc##1{\textcolor[rgb]{0.00,0.50,0.00}{##1}}}
\expandafter\def\csname PY@tok@nd\endcsname{\def\PY@tc##1{\textcolor[rgb]{0.67,0.13,1.00}{##1}}}
\expandafter\def\csname PY@tok@s\endcsname{\def\PY@tc##1{\textcolor[rgb]{0.73,0.13,0.13}{##1}}}
\expandafter\def\csname PY@tok@sd\endcsname{\let\PY@it=\textit\def\PY@tc##1{\textcolor[rgb]{0.73,0.13,0.13}{##1}}}
\expandafter\def\csname PY@tok@si\endcsname{\let\PY@bf=\textbf\def\PY@tc##1{\textcolor[rgb]{0.73,0.40,0.53}{##1}}}
\expandafter\def\csname PY@tok@se\endcsname{\let\PY@bf=\textbf\def\PY@tc##1{\textcolor[rgb]{0.73,0.40,0.13}{##1}}}
\expandafter\def\csname PY@tok@sr\endcsname{\def\PY@tc##1{\textcolor[rgb]{0.73,0.40,0.53}{##1}}}
\expandafter\def\csname PY@tok@ss\endcsname{\def\PY@tc##1{\textcolor[rgb]{0.10,0.09,0.49}{##1}}}
\expandafter\def\csname PY@tok@sx\endcsname{\def\PY@tc##1{\textcolor[rgb]{0.00,0.50,0.00}{##1}}}
\expandafter\def\csname PY@tok@m\endcsname{\def\PY@tc##1{\textcolor[rgb]{0.40,0.40,0.40}{##1}}}
\expandafter\def\csname PY@tok@gh\endcsname{\let\PY@bf=\textbf\def\PY@tc##1{\textcolor[rgb]{0.00,0.00,0.50}{##1}}}
\expandafter\def\csname PY@tok@gu\endcsname{\let\PY@bf=\textbf\def\PY@tc##1{\textcolor[rgb]{0.50,0.00,0.50}{##1}}}
\expandafter\def\csname PY@tok@gd\endcsname{\def\PY@tc##1{\textcolor[rgb]{0.63,0.00,0.00}{##1}}}
\expandafter\def\csname PY@tok@gi\endcsname{\def\PY@tc##1{\textcolor[rgb]{0.00,0.63,0.00}{##1}}}
\expandafter\def\csname PY@tok@gr\endcsname{\def\PY@tc##1{\textcolor[rgb]{1.00,0.00,0.00}{##1}}}
\expandafter\def\csname PY@tok@ge\endcsname{\let\PY@it=\textit}
\expandafter\def\csname PY@tok@gs\endcsname{\let\PY@bf=\textbf}
\expandafter\def\csname PY@tok@gp\endcsname{\let\PY@bf=\textbf\def\PY@tc##1{\textcolor[rgb]{0.00,0.00,0.50}{##1}}}
\expandafter\def\csname PY@tok@go\endcsname{\def\PY@tc##1{\textcolor[rgb]{0.53,0.53,0.53}{##1}}}
\expandafter\def\csname PY@tok@gt\endcsname{\def\PY@tc##1{\textcolor[rgb]{0.00,0.27,0.87}{##1}}}
\expandafter\def\csname PY@tok@err\endcsname{\def\PY@bc##1{\setlength{\fboxsep}{0pt}\fcolorbox[rgb]{1.00,0.00,0.00}{1,1,1}{\strut ##1}}}
\expandafter\def\csname PY@tok@kc\endcsname{\let\PY@bf=\textbf\def\PY@tc##1{\textcolor[rgb]{0.00,0.50,0.00}{##1}}}
\expandafter\def\csname PY@tok@kd\endcsname{\let\PY@bf=\textbf\def\PY@tc##1{\textcolor[rgb]{0.00,0.50,0.00}{##1}}}
\expandafter\def\csname PY@tok@kn\endcsname{\let\PY@bf=\textbf\def\PY@tc##1{\textcolor[rgb]{0.00,0.50,0.00}{##1}}}
\expandafter\def\csname PY@tok@kr\endcsname{\let\PY@bf=\textbf\def\PY@tc##1{\textcolor[rgb]{0.00,0.50,0.00}{##1}}}
\expandafter\def\csname PY@tok@bp\endcsname{\def\PY@tc##1{\textcolor[rgb]{0.00,0.50,0.00}{##1}}}
\expandafter\def\csname PY@tok@fm\endcsname{\def\PY@tc##1{\textcolor[rgb]{0.00,0.00,1.00}{##1}}}
\expandafter\def\csname PY@tok@vc\endcsname{\def\PY@tc##1{\textcolor[rgb]{0.10,0.09,0.49}{##1}}}
\expandafter\def\csname PY@tok@vg\endcsname{\def\PY@tc##1{\textcolor[rgb]{0.10,0.09,0.49}{##1}}}
\expandafter\def\csname PY@tok@vi\endcsname{\def\PY@tc##1{\textcolor[rgb]{0.10,0.09,0.49}{##1}}}
\expandafter\def\csname PY@tok@vm\endcsname{\def\PY@tc##1{\textcolor[rgb]{0.10,0.09,0.49}{##1}}}
\expandafter\def\csname PY@tok@sa\endcsname{\def\PY@tc##1{\textcolor[rgb]{0.73,0.13,0.13}{##1}}}
\expandafter\def\csname PY@tok@sb\endcsname{\def\PY@tc##1{\textcolor[rgb]{0.73,0.13,0.13}{##1}}}
\expandafter\def\csname PY@tok@sc\endcsname{\def\PY@tc##1{\textcolor[rgb]{0.73,0.13,0.13}{##1}}}
\expandafter\def\csname PY@tok@dl\endcsname{\def\PY@tc##1{\textcolor[rgb]{0.73,0.13,0.13}{##1}}}
\expandafter\def\csname PY@tok@s2\endcsname{\def\PY@tc##1{\textcolor[rgb]{0.73,0.13,0.13}{##1}}}
\expandafter\def\csname PY@tok@sh\endcsname{\def\PY@tc##1{\textcolor[rgb]{0.73,0.13,0.13}{##1}}}
\expandafter\def\csname PY@tok@s1\endcsname{\def\PY@tc##1{\textcolor[rgb]{0.73,0.13,0.13}{##1}}}
\expandafter\def\csname PY@tok@mb\endcsname{\def\PY@tc##1{\textcolor[rgb]{0.40,0.40,0.40}{##1}}}
\expandafter\def\csname PY@tok@mf\endcsname{\def\PY@tc##1{\textcolor[rgb]{0.40,0.40,0.40}{##1}}}
\expandafter\def\csname PY@tok@mh\endcsname{\def\PY@tc##1{\textcolor[rgb]{0.40,0.40,0.40}{##1}}}
\expandafter\def\csname PY@tok@mi\endcsname{\def\PY@tc##1{\textcolor[rgb]{0.40,0.40,0.40}{##1}}}
\expandafter\def\csname PY@tok@il\endcsname{\def\PY@tc##1{\textcolor[rgb]{0.40,0.40,0.40}{##1}}}
\expandafter\def\csname PY@tok@mo\endcsname{\def\PY@tc##1{\textcolor[rgb]{0.40,0.40,0.40}{##1}}}
\expandafter\def\csname PY@tok@ch\endcsname{\let\PY@it=\textit\def\PY@tc##1{\textcolor[rgb]{0.25,0.50,0.50}{##1}}}
\expandafter\def\csname PY@tok@cm\endcsname{\let\PY@it=\textit\def\PY@tc##1{\textcolor[rgb]{0.25,0.50,0.50}{##1}}}
\expandafter\def\csname PY@tok@cpf\endcsname{\let\PY@it=\textit\def\PY@tc##1{\textcolor[rgb]{0.25,0.50,0.50}{##1}}}
\expandafter\def\csname PY@tok@c1\endcsname{\let\PY@it=\textit\def\PY@tc##1{\textcolor[rgb]{0.25,0.50,0.50}{##1}}}
\expandafter\def\csname PY@tok@cs\endcsname{\let\PY@it=\textit\def\PY@tc##1{\textcolor[rgb]{0.25,0.50,0.50}{##1}}}

\def\PYZbs{\char`\\}
\def\PYZus{\char`\_}
\def\PYZob{\char`\{}
\def\PYZcb{\char`\}}
\def\PYZca{\char`\^}
\def\PYZam{\char`\&}
\def\PYZlt{\char`\<}
\def\PYZgt{\char`\>}
\def\PYZsh{\char`\#}
\def\PYZpc{\char`\%}
\def\PYZdl{\char`\$}
\def\PYZhy{\char`\-}
\def\PYZsq{\char`\'}
\def\PYZdq{\char`\"}
\def\PYZti{\char`\~}
% for compatibility with earlier versions
\def\PYZat{@}
\def\PYZlb{[}
\def\PYZrb{]}
\makeatother


    % Exact colors from NB
    \definecolor{incolor}{rgb}{0.0, 0.0, 0.5}
    \definecolor{outcolor}{rgb}{0.545, 0.0, 0.0}



    
    % Prevent overflowing lines due to hard-to-break entities
    \sloppy 
    % Setup hyperref package
    \hypersetup{
      breaklinks=true,  % so long urls are correctly broken across lines
      colorlinks=true,
      urlcolor=urlcolor,
      linkcolor=linkcolor,
      citecolor=citecolor,
      }
    % Slightly bigger margins than the latex defaults
    
    \geometry{verbose,tmargin=1in,bmargin=1in,lmargin=1in,rmargin=1in}
    
    

    \begin{document}
    
    
    \maketitle
    
    

    
    \section{Series 5}\label{series-5}

    \subsection{Aufgabe 5.1}\label{aufgabe-5.1}

\begin{quote}
Normalverteile, t-verteilte, chiquadrat-verteile Zufallszahlen,
Freiheitsgrade, QQ-Plot
\end{quote}

    \begin{center}\rule{0.5\linewidth}{\linethickness}\end{center}

\textbf{a.) Normalverteilte Zufallszahlen simulieren und mit Normalplot
betrachten.} \(n\) wird als Parameter (size) der Methode rvs übergeben.
\(n\) ist die Anzahl standard-normalverteilte Zufallszahlen. Wenn man
die Simulation wiederholt, ändert sich die Streuung jedoch immer um die
Winkelhalbierende des QQ-Plots. Je mehr Zufallszahlen generiert werden,
desto stärker normalverteilt wird der QQ-Plot. (Bei einer uniformen
Verteilung bewirken mehr Zufallszahlen, dass die Gerade immer stärker
uniform verteilt wird. Die Summe aller Zufallszahlen wird immer stärker
normalverteilt, je mehr Zufallszahlen dazukommen)

    \begin{Verbatim}[commandchars=\\\{\}]
{\color{incolor}In [{\color{incolor}22}]:} \PY{k+kn}{import} \PY{n+nn}{matplotlib}\PY{n+nn}{.}\PY{n+nn}{pyplot} \PY{k}{as} \PY{n+nn}{plt}
         \PY{k+kn}{import} \PY{n+nn}{scipy}\PY{n+nn}{.}\PY{n+nn}{stats} \PY{k}{as} \PY{n+nn}{st}
         
         \PY{n}{plt}\PY{o}{.}\PY{n}{subplot}\PY{p}{(}\PY{l+m+mi}{2}\PY{p}{,}\PY{l+m+mi}{2}\PY{p}{,}\PY{l+m+mi}{1}\PY{p}{)}
         \PY{n}{x} \PY{o}{=} \PY{n}{st}\PY{o}{.}\PY{n}{norm}\PY{o}{.}\PY{n}{rvs}\PY{p}{(}\PY{n}{size}\PY{o}{=}\PY{l+m+mi}{10}\PY{p}{)}
         \PY{n}{st}\PY{o}{.}\PY{n}{probplot}\PY{p}{(}\PY{n}{x}\PY{p}{,} \PY{n}{plot}\PY{o}{=}\PY{n}{plt}\PY{p}{)}
         \PY{n}{plt}\PY{o}{.}\PY{n}{title}\PY{p}{(}\PY{l+s+s2}{\PYZdq{}}\PY{l+s+s2}{n = 10}\PY{l+s+s2}{\PYZdq{}}\PY{p}{)}
         
         \PY{n}{plt}\PY{o}{.}\PY{n}{subplot}\PY{p}{(}\PY{l+m+mi}{2}\PY{p}{,}\PY{l+m+mi}{2}\PY{p}{,}\PY{l+m+mi}{2}\PY{p}{)}
         \PY{n}{x} \PY{o}{=} \PY{n}{st}\PY{o}{.}\PY{n}{norm}\PY{o}{.}\PY{n}{rvs}\PY{p}{(}\PY{n}{size}\PY{o}{=}\PY{l+m+mi}{20}\PY{p}{)}
         \PY{n}{st}\PY{o}{.}\PY{n}{probplot}\PY{p}{(}\PY{n}{x}\PY{p}{,} \PY{n}{plot}\PY{o}{=}\PY{n}{plt}\PY{p}{)}
         \PY{n}{plt}\PY{o}{.}\PY{n}{title}\PY{p}{(}\PY{l+s+s2}{\PYZdq{}}\PY{l+s+s2}{n = 20}\PY{l+s+s2}{\PYZdq{}}\PY{p}{)}
         
         \PY{n}{plt}\PY{o}{.}\PY{n}{subplot}\PY{p}{(}\PY{l+m+mi}{2}\PY{p}{,}\PY{l+m+mi}{2}\PY{p}{,}\PY{l+m+mi}{3}\PY{p}{)}
         \PY{n}{x} \PY{o}{=} \PY{n}{st}\PY{o}{.}\PY{n}{norm}\PY{o}{.}\PY{n}{rvs}\PY{p}{(}\PY{n}{size}\PY{o}{=}\PY{l+m+mi}{50}\PY{p}{)}
         \PY{n}{st}\PY{o}{.}\PY{n}{probplot}\PY{p}{(}\PY{n}{x}\PY{p}{,} \PY{n}{plot}\PY{o}{=}\PY{n}{plt}\PY{p}{)}
         \PY{n}{plt}\PY{o}{.}\PY{n}{title}\PY{p}{(}\PY{l+s+s2}{\PYZdq{}}\PY{l+s+s2}{n = 50}\PY{l+s+s2}{\PYZdq{}}\PY{p}{)}
         
         \PY{n}{plt}\PY{o}{.}\PY{n}{subplot}\PY{p}{(}\PY{l+m+mi}{2}\PY{p}{,}\PY{l+m+mi}{2}\PY{p}{,}\PY{l+m+mi}{4}\PY{p}{)}
         \PY{n}{x} \PY{o}{=} \PY{n}{st}\PY{o}{.}\PY{n}{norm}\PY{o}{.}\PY{n}{rvs}\PY{p}{(}\PY{n}{size}\PY{o}{=}\PY{l+m+mi}{100}\PY{p}{)}
         \PY{n}{st}\PY{o}{.}\PY{n}{probplot}\PY{p}{(}\PY{n}{x}\PY{p}{,} \PY{n}{plot}\PY{o}{=}\PY{n}{plt}\PY{p}{)}
         \PY{n}{plt}\PY{o}{.}\PY{n}{title}\PY{p}{(}\PY{l+s+s2}{\PYZdq{}}\PY{l+s+s2}{n = 100}\PY{l+s+s2}{\PYZdq{}}\PY{p}{)}
         
         \PY{n}{plt}\PY{o}{.}\PY{n}{tight\PYZus{}layout}\PY{p}{(}\PY{p}{)}
         \PY{n}{plt}\PY{o}{.}\PY{n}{show}\PY{p}{(}\PY{p}{)}
\end{Verbatim}


    \begin{center}
    \adjustimage{max size={0.9\linewidth}{0.9\paperheight}}{output_3_0.png}
    \end{center}
    { \hspace*{\fill} \\}
    
    \begin{center}\rule{0.5\linewidth}{\linethickness}\end{center}

\textbf{b.) Langschwänzige Verteilung: t-verteilte Zufallszahlen
simulieren mit Freiheitsgraden.} Die t-Verteilung näher sich der
standard-Normalverteilung je mehr Zufallszahlen und je mehr
Freiheitsgrade existieren. Das heisst die t-Verteilung mit 100
Zufallszahlen und 20 Freiheitsgraden nähert sich am meisten einer
standard-Normalverteilung.

    \begin{Verbatim}[commandchars=\\\{\}]
{\color{incolor}In [{\color{incolor}23}]:} \PY{k+kn}{import} \PY{n+nn}{matplotlib}\PY{n+nn}{.}\PY{n+nn}{pyplot} \PY{k}{as} \PY{n+nn}{plt}
         \PY{k+kn}{import} \PY{n+nn}{scipy}\PY{n+nn}{.}\PY{n+nn}{stats} \PY{k}{as} \PY{n+nn}{st}
         
         \PY{k}{for} \PY{n}{i} \PY{o+ow}{in} \PY{n+nb}{range}\PY{p}{(}\PY{l+m+mi}{1}\PY{p}{,} \PY{l+m+mi}{4}\PY{p}{)}\PY{p}{:}
             \PY{n}{plt}\PY{o}{.}\PY{n}{subplot}\PY{p}{(}\PY{l+m+mi}{1}\PY{p}{,} \PY{l+m+mi}{3}\PY{p}{,} \PY{n}{i}\PY{p}{)}
             \PY{n}{x} \PY{o}{=} \PY{n}{st}\PY{o}{.}\PY{n}{t}\PY{o}{.}\PY{n}{rvs}\PY{p}{(}\PY{n}{size}\PY{o}{=}\PY{l+m+mi}{20}\PY{p}{,} \PY{n}{df}\PY{o}{=}\PY{l+m+mi}{20}\PY{p}{)}
             \PY{n}{st}\PY{o}{.}\PY{n}{probplot}\PY{p}{(}\PY{n}{x}\PY{p}{,} \PY{n}{plot}\PY{o}{=}\PY{n}{plt}\PY{p}{)}
             \PY{n}{plt}\PY{o}{.}\PY{n}{title}\PY{p}{(}\PY{l+s+s2}{\PYZdq{}}\PY{l+s+s2}{n = 20, df = 20}\PY{l+s+s2}{\PYZdq{}}\PY{p}{)}
             
         \PY{n}{plt}\PY{o}{.}\PY{n}{tight\PYZus{}layout}\PY{p}{(}\PY{p}{)}
         \PY{n}{plt}\PY{o}{.}\PY{n}{show}\PY{p}{(}\PY{p}{)}
         
         \PY{k}{for} \PY{n}{i} \PY{o+ow}{in} \PY{n+nb}{range}\PY{p}{(}\PY{l+m+mi}{1}\PY{p}{,} \PY{l+m+mi}{4}\PY{p}{)}\PY{p}{:}
             \PY{n}{plt}\PY{o}{.}\PY{n}{subplot}\PY{p}{(}\PY{l+m+mi}{1}\PY{p}{,} \PY{l+m+mi}{3}\PY{p}{,} \PY{n}{i}\PY{p}{)}
             \PY{n}{x} \PY{o}{=} \PY{n}{st}\PY{o}{.}\PY{n}{t}\PY{o}{.}\PY{n}{rvs}\PY{p}{(}\PY{n}{size}\PY{o}{=}\PY{l+m+mi}{100}\PY{p}{,} \PY{n}{df}\PY{o}{=}\PY{l+m+mi}{20}\PY{p}{)}
             \PY{n}{st}\PY{o}{.}\PY{n}{probplot}\PY{p}{(}\PY{n}{x}\PY{p}{,} \PY{n}{plot}\PY{o}{=}\PY{n}{plt}\PY{p}{)}
             \PY{n}{plt}\PY{o}{.}\PY{n}{title}\PY{p}{(}\PY{l+s+s2}{\PYZdq{}}\PY{l+s+s2}{n = 100, df = 20}\PY{l+s+s2}{\PYZdq{}}\PY{p}{)}
         
         \PY{n}{plt}\PY{o}{.}\PY{n}{tight\PYZus{}layout}\PY{p}{(}\PY{p}{)}
         \PY{n}{plt}\PY{o}{.}\PY{n}{show}\PY{p}{(}\PY{p}{)}
         
         \PY{k}{for} \PY{n}{i} \PY{o+ow}{in} \PY{n+nb}{range}\PY{p}{(}\PY{l+m+mi}{1}\PY{p}{,} \PY{l+m+mi}{4}\PY{p}{)}\PY{p}{:}
             \PY{n}{plt}\PY{o}{.}\PY{n}{subplot}\PY{p}{(}\PY{l+m+mi}{1}\PY{p}{,} \PY{l+m+mi}{3}\PY{p}{,} \PY{n}{i}\PY{p}{)}
             \PY{n}{x} \PY{o}{=} \PY{n}{st}\PY{o}{.}\PY{n}{t}\PY{o}{.}\PY{n}{rvs}\PY{p}{(}\PY{n}{size}\PY{o}{=}\PY{l+m+mi}{20}\PY{p}{,} \PY{n}{df}\PY{o}{=}\PY{l+m+mi}{7}\PY{p}{)}
             \PY{n}{st}\PY{o}{.}\PY{n}{probplot}\PY{p}{(}\PY{n}{x}\PY{p}{,} \PY{n}{plot}\PY{o}{=}\PY{n}{plt}\PY{p}{)}
             \PY{n}{plt}\PY{o}{.}\PY{n}{title}\PY{p}{(}\PY{l+s+s2}{\PYZdq{}}\PY{l+s+s2}{n = 20, df = 7}\PY{l+s+s2}{\PYZdq{}}\PY{p}{)}
             
         \PY{n}{plt}\PY{o}{.}\PY{n}{tight\PYZus{}layout}\PY{p}{(}\PY{p}{)}
         \PY{n}{plt}\PY{o}{.}\PY{n}{show}\PY{p}{(}\PY{p}{)}
         
         \PY{k}{for} \PY{n}{i} \PY{o+ow}{in} \PY{n+nb}{range}\PY{p}{(}\PY{l+m+mi}{1}\PY{p}{,} \PY{l+m+mi}{4}\PY{p}{)}\PY{p}{:}
             \PY{n}{plt}\PY{o}{.}\PY{n}{subplot}\PY{p}{(}\PY{l+m+mi}{1}\PY{p}{,} \PY{l+m+mi}{3}\PY{p}{,} \PY{n}{i}\PY{p}{)}
             \PY{n}{x} \PY{o}{=} \PY{n}{st}\PY{o}{.}\PY{n}{t}\PY{o}{.}\PY{n}{rvs}\PY{p}{(}\PY{n}{size}\PY{o}{=}\PY{l+m+mi}{100}\PY{p}{,} \PY{n}{df}\PY{o}{=}\PY{l+m+mi}{7}\PY{p}{)}
             \PY{n}{st}\PY{o}{.}\PY{n}{probplot}\PY{p}{(}\PY{n}{x}\PY{p}{,} \PY{n}{plot}\PY{o}{=}\PY{n}{plt}\PY{p}{)}
             \PY{n}{plt}\PY{o}{.}\PY{n}{title}\PY{p}{(}\PY{l+s+s2}{\PYZdq{}}\PY{l+s+s2}{n = 100, df = 7}\PY{l+s+s2}{\PYZdq{}}\PY{p}{)}
             
         \PY{n}{plt}\PY{o}{.}\PY{n}{tight\PYZus{}layout}\PY{p}{(}\PY{p}{)}
         \PY{n}{plt}\PY{o}{.}\PY{n}{show}\PY{p}{(}\PY{p}{)}
         
         \PY{k}{for} \PY{n}{i} \PY{o+ow}{in} \PY{n+nb}{range}\PY{p}{(}\PY{l+m+mi}{1}\PY{p}{,} \PY{l+m+mi}{4}\PY{p}{)}\PY{p}{:}
             \PY{n}{plt}\PY{o}{.}\PY{n}{subplot}\PY{p}{(}\PY{l+m+mi}{1}\PY{p}{,} \PY{l+m+mi}{3}\PY{p}{,} \PY{n}{i}\PY{p}{)}
             \PY{n}{x} \PY{o}{=} \PY{n}{st}\PY{o}{.}\PY{n}{t}\PY{o}{.}\PY{n}{rvs}\PY{p}{(}\PY{n}{size}\PY{o}{=}\PY{l+m+mi}{20}\PY{p}{,} \PY{n}{df}\PY{o}{=}\PY{l+m+mi}{3}\PY{p}{)}
             \PY{n}{st}\PY{o}{.}\PY{n}{probplot}\PY{p}{(}\PY{n}{x}\PY{p}{,} \PY{n}{plot}\PY{o}{=}\PY{n}{plt}\PY{p}{)}
             \PY{n}{plt}\PY{o}{.}\PY{n}{title}\PY{p}{(}\PY{l+s+s2}{\PYZdq{}}\PY{l+s+s2}{n = 20, df = 3}\PY{l+s+s2}{\PYZdq{}}\PY{p}{)}
             
         \PY{n}{plt}\PY{o}{.}\PY{n}{tight\PYZus{}layout}\PY{p}{(}\PY{p}{)}
         \PY{n}{plt}\PY{o}{.}\PY{n}{show}\PY{p}{(}\PY{p}{)}
         
         \PY{k}{for} \PY{n}{i} \PY{o+ow}{in} \PY{n+nb}{range}\PY{p}{(}\PY{l+m+mi}{1}\PY{p}{,} \PY{l+m+mi}{4}\PY{p}{)}\PY{p}{:}
             \PY{n}{plt}\PY{o}{.}\PY{n}{subplot}\PY{p}{(}\PY{l+m+mi}{1}\PY{p}{,} \PY{l+m+mi}{3}\PY{p}{,} \PY{n}{i}\PY{p}{)}
             \PY{n}{x} \PY{o}{=} \PY{n}{st}\PY{o}{.}\PY{n}{t}\PY{o}{.}\PY{n}{rvs}\PY{p}{(}\PY{n}{size}\PY{o}{=}\PY{l+m+mi}{100}\PY{p}{,} \PY{n}{df}\PY{o}{=}\PY{l+m+mi}{3}\PY{p}{)}
             \PY{n}{st}\PY{o}{.}\PY{n}{probplot}\PY{p}{(}\PY{n}{x}\PY{p}{,} \PY{n}{plot}\PY{o}{=}\PY{n}{plt}\PY{p}{)}
             \PY{n}{plt}\PY{o}{.}\PY{n}{title}\PY{p}{(}\PY{l+s+s2}{\PYZdq{}}\PY{l+s+s2}{n = 100, df = 3}\PY{l+s+s2}{\PYZdq{}}\PY{p}{)}
             
         \PY{n}{plt}\PY{o}{.}\PY{n}{tight\PYZus{}layout}\PY{p}{(}\PY{p}{)}
         \PY{n}{plt}\PY{o}{.}\PY{n}{show}\PY{p}{(}\PY{p}{)}
\end{Verbatim}


    \begin{center}
    \adjustimage{max size={0.9\linewidth}{0.9\paperheight}}{output_5_0.png}
    \end{center}
    { \hspace*{\fill} \\}
    
    \begin{center}
    \adjustimage{max size={0.9\linewidth}{0.9\paperheight}}{output_5_1.png}
    \end{center}
    { \hspace*{\fill} \\}
    
    \begin{center}
    \adjustimage{max size={0.9\linewidth}{0.9\paperheight}}{output_5_2.png}
    \end{center}
    { \hspace*{\fill} \\}
    
    \begin{center}
    \adjustimage{max size={0.9\linewidth}{0.9\paperheight}}{output_5_3.png}
    \end{center}
    { \hspace*{\fill} \\}
    
    \begin{center}
    \adjustimage{max size={0.9\linewidth}{0.9\paperheight}}{output_5_4.png}
    \end{center}
    { \hspace*{\fill} \\}
    
    \begin{center}
    \adjustimage{max size={0.9\linewidth}{0.9\paperheight}}{output_5_5.png}
    \end{center}
    { \hspace*{\fill} \\}
    
    \begin{center}\rule{0.5\linewidth}{\linethickness}\end{center}

\textbf{c.) Schiefe Verteilung: chiquadrat-verteilte Zufallszahlen mit
Freiheitsgraden.} Je mehr Freiheitsgrade die chiquadrat-Verteilung
besitzt desto mehr formt sie sich zu einem Normalplot. Hier spielt die
Anzahl Zufallsvariablen weniger bis gar keine Rolle im Gegensatz zu
einer t-Verteilung.

    \begin{Verbatim}[commandchars=\\\{\}]
{\color{incolor}In [{\color{incolor}6}]:} \PY{k+kn}{import} \PY{n+nn}{matplotlib}\PY{n+nn}{.}\PY{n+nn}{pyplot} \PY{k}{as} \PY{n+nn}{plt}
        \PY{k+kn}{import} \PY{n+nn}{scipy}\PY{n+nn}{.}\PY{n+nn}{stats} \PY{k}{as} \PY{n+nn}{st}
        
        \PY{k}{for} \PY{n}{i} \PY{o+ow}{in} \PY{n+nb}{range}\PY{p}{(}\PY{l+m+mi}{1}\PY{p}{,} \PY{l+m+mi}{4}\PY{p}{)}\PY{p}{:}
            \PY{n}{plt}\PY{o}{.}\PY{n}{subplot}\PY{p}{(}\PY{l+m+mi}{1}\PY{p}{,} \PY{l+m+mi}{3}\PY{p}{,} \PY{n}{i}\PY{p}{)}
            \PY{n}{x} \PY{o}{=} \PY{n}{st}\PY{o}{.}\PY{n}{chi2}\PY{o}{.}\PY{n}{rvs}\PY{p}{(}\PY{n}{size}\PY{o}{=} \PY{l+m+mi}{20}\PY{p}{,} \PY{n}{df}\PY{o}{=}\PY{l+m+mi}{20}\PY{p}{)}
            \PY{n}{st}\PY{o}{.}\PY{n}{probplot}\PY{p}{(}\PY{n}{x}\PY{p}{,} \PY{n}{plot}\PY{o}{=}\PY{n}{plt}\PY{p}{)}
            \PY{n}{plt}\PY{o}{.}\PY{n}{title}\PY{p}{(}\PY{l+s+s2}{\PYZdq{}}\PY{l+s+s2}{n = 20, df = 20}\PY{l+s+s2}{\PYZdq{}}\PY{p}{)}
            
        \PY{n}{plt}\PY{o}{.}\PY{n}{tight\PYZus{}layout}\PY{p}{(}\PY{p}{)}
        \PY{n}{plt}\PY{o}{.}\PY{n}{show}\PY{p}{(}\PY{p}{)}
        
        \PY{k+kn}{import} \PY{n+nn}{matplotlib}\PY{n+nn}{.}\PY{n+nn}{pyplot} \PY{k}{as} \PY{n+nn}{plt}
        \PY{k+kn}{import} \PY{n+nn}{scipy}\PY{n+nn}{.}\PY{n+nn}{stats} \PY{k}{as} \PY{n+nn}{st}
        
        \PY{k}{for} \PY{n}{i} \PY{o+ow}{in} \PY{n+nb}{range}\PY{p}{(}\PY{l+m+mi}{1}\PY{p}{,} \PY{l+m+mi}{4}\PY{p}{)}\PY{p}{:}
            \PY{n}{plt}\PY{o}{.}\PY{n}{subplot}\PY{p}{(}\PY{l+m+mi}{1}\PY{p}{,} \PY{l+m+mi}{3}\PY{p}{,} \PY{n}{i}\PY{p}{)}
            \PY{n}{x} \PY{o}{=} \PY{n}{st}\PY{o}{.}\PY{n}{chi2}\PY{o}{.}\PY{n}{rvs}\PY{p}{(}\PY{n}{size}\PY{o}{=} \PY{l+m+mi}{100}\PY{p}{,} \PY{n}{df}\PY{o}{=}\PY{l+m+mi}{20}\PY{p}{)}
            \PY{n}{st}\PY{o}{.}\PY{n}{probplot}\PY{p}{(}\PY{n}{x}\PY{p}{,} \PY{n}{plot}\PY{o}{=}\PY{n}{plt}\PY{p}{)}
            \PY{n}{plt}\PY{o}{.}\PY{n}{title}\PY{p}{(}\PY{l+s+s2}{\PYZdq{}}\PY{l+s+s2}{n = 100, df = 20}\PY{l+s+s2}{\PYZdq{}}\PY{p}{)}
            
        \PY{n}{plt}\PY{o}{.}\PY{n}{tight\PYZus{}layout}\PY{p}{(}\PY{p}{)}
        \PY{n}{plt}\PY{o}{.}\PY{n}{show}\PY{p}{(}\PY{p}{)}
        
        \PY{k+kn}{import} \PY{n+nn}{matplotlib}\PY{n+nn}{.}\PY{n+nn}{pyplot} \PY{k}{as} \PY{n+nn}{plt}
        \PY{k+kn}{import} \PY{n+nn}{scipy}\PY{n+nn}{.}\PY{n+nn}{stats} \PY{k}{as} \PY{n+nn}{st}
        
        \PY{k}{for} \PY{n}{i} \PY{o+ow}{in} \PY{n+nb}{range}\PY{p}{(}\PY{l+m+mi}{1}\PY{p}{,} \PY{l+m+mi}{4}\PY{p}{)}\PY{p}{:}
            \PY{n}{plt}\PY{o}{.}\PY{n}{subplot}\PY{p}{(}\PY{l+m+mi}{1}\PY{p}{,} \PY{l+m+mi}{3}\PY{p}{,} \PY{n}{i}\PY{p}{)}
            \PY{n}{x} \PY{o}{=} \PY{n}{st}\PY{o}{.}\PY{n}{chi2}\PY{o}{.}\PY{n}{rvs}\PY{p}{(}\PY{n}{size}\PY{o}{=} \PY{l+m+mi}{20}\PY{p}{,} \PY{n}{df}\PY{o}{=}\PY{l+m+mi}{1}\PY{p}{)}
            \PY{n}{st}\PY{o}{.}\PY{n}{probplot}\PY{p}{(}\PY{n}{x}\PY{p}{,} \PY{n}{plot}\PY{o}{=}\PY{n}{plt}\PY{p}{)}
            \PY{n}{plt}\PY{o}{.}\PY{n}{title}\PY{p}{(}\PY{l+s+s2}{\PYZdq{}}\PY{l+s+s2}{n = 20, df = 1}\PY{l+s+s2}{\PYZdq{}}\PY{p}{)}
            
        \PY{n}{plt}\PY{o}{.}\PY{n}{tight\PYZus{}layout}\PY{p}{(}\PY{p}{)}
        \PY{n}{plt}\PY{o}{.}\PY{n}{show}\PY{p}{(}\PY{p}{)}
        
        \PY{k+kn}{import} \PY{n+nn}{matplotlib}\PY{n+nn}{.}\PY{n+nn}{pyplot} \PY{k}{as} \PY{n+nn}{plt}
        \PY{k+kn}{import} \PY{n+nn}{scipy}\PY{n+nn}{.}\PY{n+nn}{stats} \PY{k}{as} \PY{n+nn}{st}
        
        \PY{k}{for} \PY{n}{i} \PY{o+ow}{in} \PY{n+nb}{range}\PY{p}{(}\PY{l+m+mi}{1}\PY{p}{,} \PY{l+m+mi}{4}\PY{p}{)}\PY{p}{:}
            \PY{n}{plt}\PY{o}{.}\PY{n}{subplot}\PY{p}{(}\PY{l+m+mi}{1}\PY{p}{,} \PY{l+m+mi}{3}\PY{p}{,} \PY{n}{i}\PY{p}{)}
            \PY{n}{x} \PY{o}{=} \PY{n}{st}\PY{o}{.}\PY{n}{chi2}\PY{o}{.}\PY{n}{rvs}\PY{p}{(}\PY{n}{size}\PY{o}{=} \PY{l+m+mi}{100}\PY{p}{,} \PY{n}{df}\PY{o}{=}\PY{l+m+mi}{1}\PY{p}{)}
            \PY{n}{st}\PY{o}{.}\PY{n}{probplot}\PY{p}{(}\PY{n}{x}\PY{p}{,} \PY{n}{plot}\PY{o}{=}\PY{n}{plt}\PY{p}{)}
            \PY{n}{plt}\PY{o}{.}\PY{n}{title}\PY{p}{(}\PY{l+s+s2}{\PYZdq{}}\PY{l+s+s2}{n = 100, df = 1}\PY{l+s+s2}{\PYZdq{}}\PY{p}{)}
            
        \PY{n}{plt}\PY{o}{.}\PY{n}{tight\PYZus{}layout}\PY{p}{(}\PY{p}{)}
        \PY{n}{plt}\PY{o}{.}\PY{n}{show}\PY{p}{(}\PY{p}{)}
\end{Verbatim}


    \begin{center}
    \adjustimage{max size={0.9\linewidth}{0.9\paperheight}}{output_7_0.png}
    \end{center}
    { \hspace*{\fill} \\}
    
    \begin{center}
    \adjustimage{max size={0.9\linewidth}{0.9\paperheight}}{output_7_1.png}
    \end{center}
    { \hspace*{\fill} \\}
    
    \begin{center}
    \adjustimage{max size={0.9\linewidth}{0.9\paperheight}}{output_7_2.png}
    \end{center}
    { \hspace*{\fill} \\}
    
    \begin{center}
    \adjustimage{max size={0.9\linewidth}{0.9\paperheight}}{output_7_3.png}
    \end{center}
    { \hspace*{\fill} \\}
    
    \subsection{Aufgabe 5.2}\label{aufgabe-5.2}

\begin{quote}
Zentraler Grenzwertsatz, Histogramm, Normalplot
\end{quote}

    In dieser Aufgabe untersuchen Sie die \textbf{Wirkung des Zentralen
Grenzwertsatzes mittels Simulation}. Gehen Sie von einer
Zufallsvariablen \(X\) aus, die folgendermassen verteilt ist: die Werte
0, 10 und 11 werden je mit einer Wahrscheinlichkeit \(\frac{1}{3}\)
angenommen. Das heisst, dass jede Zahl gleichwahrscheinlich angenommen
werden kann.

    Wir simulieren nun die Verteilung von \(X\) sowie die Verteilung des
Mittelwerts \(\overline{X}_n\) von mehreren \(X\).

\begin{center}\rule{0.5\linewidth}{\linethickness}\end{center}

\textbf{a.) Die Verteilung von X mittels eines Histogramms darstellen}

\begin{enumerate}
\def\labelenumi{\arabic{enumi}.}
\tightlist
\item
  mögliche Werte von X definieren
\item
  X simulieren mit Series(np.random.choice(werte, size, replace=True)
\item
  Subplot für Histogramm
\item
  Histogramm erstellen
\item
  Subplot für Normalplot
\item
  Normalplot erstellen
\end{enumerate}

    \begin{Verbatim}[commandchars=\\\{\}]
{\color{incolor}In [{\color{incolor}1}]:} \PY{k+kn}{import} \PY{n+nn}{matplotlib}\PY{n+nn}{.}\PY{n+nn}{pyplot} \PY{k}{as} \PY{n+nn}{plt}
        \PY{k+kn}{import} \PY{n+nn}{numpy} \PY{k}{as} \PY{n+nn}{np}
        \PY{k+kn}{from} \PY{n+nn}{pandas} \PY{k}{import} \PY{n}{Series}\PY{p}{,} \PY{n}{DataFrame}
        \PY{k+kn}{import} \PY{n+nn}{scipy}\PY{n+nn}{.}\PY{n+nn}{stats} \PY{k}{as} \PY{n+nn}{st}
        
        \PY{c+c1}{\PYZsh{} 1. mögliche Werte von X definieren}
        \PY{n}{werte} \PY{o}{=} \PY{n}{np}\PY{o}{.}\PY{n}{array}\PY{p}{(}\PY{p}{[}\PY{l+m+mi}{0}\PY{p}{,} \PY{l+m+mi}{10}\PY{p}{,} \PY{l+m+mi}{11}\PY{p}{]}\PY{p}{)}
        
        \PY{c+c1}{\PYZsh{} 2. X simulieren mit Series(np.random.choice())}
        \PY{n}{sim} \PY{o}{=} \PY{n}{Series}\PY{p}{(}\PY{n}{np}\PY{o}{.}\PY{n}{random}\PY{o}{.}\PY{n}{choice}\PY{p}{(}\PY{n}{werte}\PY{p}{,} \PY{n}{size}\PY{o}{=}\PY{l+m+mi}{1000}\PY{p}{,} \PY{n}{replace}\PY{o}{=}\PY{k+kc}{True}\PY{p}{)}\PY{p}{)}
        
        \PY{c+c1}{\PYZsh{} 3. Subplot für Histogramm}
        \PY{n}{plt}\PY{o}{.}\PY{n}{subplot}\PY{p}{(}\PY{l+m+mi}{1}\PY{p}{,} \PY{l+m+mi}{2}\PY{p}{,} \PY{l+m+mi}{1}\PY{p}{)}
        
        \PY{c+c1}{\PYZsh{} 4. Histogramm erstellen}
        \PY{n}{sim}\PY{o}{.}\PY{n}{hist}\PY{p}{(}\PY{n}{bins}\PY{o}{=}\PY{p}{[}\PY{l+m+mi}{0}\PY{p}{,} \PY{l+m+mi}{1}\PY{p}{,} \PY{l+m+mi}{10}\PY{p}{,} \PY{l+m+mi}{11}\PY{p}{,} \PY{l+m+mi}{12}\PY{p}{]}\PY{p}{,} \PY{n}{edgecolor}\PY{o}{=}\PY{l+s+s2}{\PYZdq{}}\PY{l+s+s2}{white}\PY{l+s+s2}{\PYZdq{}}\PY{p}{)}
        \PY{n}{plt}\PY{o}{.}\PY{n}{title}\PY{p}{(}\PY{l+s+s2}{\PYZdq{}}\PY{l+s+s2}{Histogramm}\PY{l+s+s2}{\PYZdq{}}\PY{p}{)}
        
        \PY{c+c1}{\PYZsh{} 5. Subplot für Normalplot}
        \PY{n}{plt}\PY{o}{.}\PY{n}{subplot}\PY{p}{(}\PY{l+m+mi}{1}\PY{p}{,} \PY{l+m+mi}{2}\PY{p}{,} \PY{l+m+mi}{2}\PY{p}{)}
        
        \PY{c+c1}{\PYZsh{} 6. Normalplot erstellen}
        \PY{n}{st}\PY{o}{.}\PY{n}{probplot}\PY{p}{(}\PY{n}{sim}\PY{p}{,} \PY{n}{plot}\PY{o}{=}\PY{n}{plt}\PY{p}{)}
        \PY{n}{plt}\PY{o}{.}\PY{n}{title}\PY{p}{(}\PY{l+s+s2}{\PYZdq{}}\PY{l+s+s2}{Normal Q\PYZhy{}Q Plot}\PY{l+s+s2}{\PYZdq{}}\PY{p}{)}
        \PY{n}{plt}\PY{o}{.}\PY{n}{tight\PYZus{}layout}\PY{p}{(}\PY{p}{)}
        \PY{n}{plt}\PY{o}{.}\PY{n}{show}\PY{p}{(}\PY{p}{)}
\end{Verbatim}


    
    \begin{verbatim}
<matplotlib.figure.Figure at 0x106fa0c50>
    \end{verbatim}

    
    \begin{center}\rule{0.5\linewidth}{\linethickness}\end{center}

\textbf{b.) Wir simulieren nun
\(\overline{X}_5 = \frac{X_1+X_2+X_3+X_4+X_5}{5}\), wobei die \(X_i\)
die gleiche Verteilung haben wie \(X\) und unabhängig sind.} Stellen Sie
die Verteilung von \(\overline{X}_5\) anhand von 1000 Realisierungen von
\(\overline{X}_5\) dar, und vergleichen Sie mit der Normalverteilung.

\begin{enumerate}
\def\labelenumi{\arabic{enumi}.}
\tightlist
\item
  \(X_1\), ..., \(X_n\) simulieren und in einer \(n\)-spaltigen Matrix
  (mit 1000 Zeilen) anordnen
\item
  In jeder Matrixzeile Mittelwert berechnen
\item
  Histogramm erstellen
\item
  QQ-Normalplot erstellen
\end{enumerate}

Je mehr Beobachtungen wir durchführen, sprich je höher \(n\) ist, desto
mehr wird das Histogramm und der QQ-Plot normalverteilt. Für das
Verständnis:

\begin{itemize}
\tightlist
\item
  Es kann die Zahlen 0, 10 oder 11 annehmen.
\item
  Bei einem ersten Vorgang mit 1000 Realisierungen gab es zufällig 200
  mal 0, 500 mal 10 und 300 mal 11.
\item
  Der Mittelwert dieser Zahlen ist also
  \(\frac{200*0 + 500*10 + 300*11}{1000} = 8.3\)
\item
  Bei einem weiteren Vorgang mit 1000 weiteren Realisierungen gab es
  zufällig 300 mal 0, 100 mal 10 und 600 mal 11.
\item
  Dies gibt wieder einen neuen Wert (7.6)
\item
  Die Hauptaussage ist, je mehr Wiederholungen dieses Vorgangs man
  vornimmt, desto normalverteilter wird die Streuung dieser Mittelwerte
\end{itemize}

\begin{quote}
5 Vorgänge:
\(\frac{\frac{200*0+500*10+300*11}{1000}+\frac{200*0+700*10+100*11}{1000}+\frac{800*0+100*10+100*11}{1000}+\frac{500*0+500*10+0*11}{1000}+\frac{400*0+100*10+500*11}{1000}}{5} = 6\)
\end{quote}

    \begin{Verbatim}[commandchars=\\\{\}]
{\color{incolor}In [{\color{incolor}43}]:} \PY{c+c1}{\PYZsh{} Anzahl Beobachtungen definieren}
         \PY{n}{n} \PY{o}{=} \PY{l+m+mi}{5}
         
         \PY{c+c1}{\PYZsh{} 1. X\PYZus{}1, ..., X\PYZus{}n simulieren und in einer n\PYZhy{}spaltigen Matrix (mit 1000 Zeilen) anordnen}
         \PY{n}{sim} \PY{o}{=} \PY{n}{Series}\PY{p}{(}\PY{n}{np}\PY{o}{.}\PY{n}{random}\PY{o}{.}\PY{n}{choice}\PY{p}{(}\PY{n}{werte}\PY{p}{,} \PY{n}{size}\PY{o}{=}\PY{n}{n}\PY{o}{*}\PY{l+m+mi}{1000}\PY{p}{,} \PY{n}{replace}\PY{o}{=}\PY{k+kc}{True}\PY{p}{)}\PY{p}{)}
         \PY{n}{sim} \PY{o}{=} \PY{n}{DataFrame}\PY{p}{(}\PY{n}{np}\PY{o}{.}\PY{n}{reshape}\PY{p}{(}\PY{n}{sim}\PY{p}{,} \PY{p}{(}\PY{n}{n}\PY{p}{,} \PY{l+m+mi}{1000}\PY{p}{)}\PY{p}{)}\PY{p}{)}
         
         \PY{c+c1}{\PYZsh{} 2. In jeder Matrixzeile Mittelwert berechnen}
         \PY{n}{sim\PYZus{}mean} \PY{o}{=} \PY{n}{sim}\PY{o}{.}\PY{n}{mean}\PY{p}{(}\PY{p}{)}
         
         \PY{c+c1}{\PYZsh{} 3. Histogramm erstellen}
         \PY{n}{plt}\PY{o}{.}\PY{n}{subplot}\PY{p}{(}\PY{l+m+mi}{2}\PY{p}{,}\PY{l+m+mi}{2}\PY{p}{,}\PY{l+m+mi}{1}\PY{p}{)}
         \PY{n}{sim\PYZus{}mean}\PY{o}{.}\PY{n}{hist}\PY{p}{(}\PY{n}{edgecolor}\PY{o}{=}\PY{l+s+s2}{\PYZdq{}}\PY{l+s+s2}{white}\PY{l+s+s2}{\PYZdq{}}\PY{p}{)}
         \PY{n}{plt}\PY{o}{.}\PY{n}{title}\PY{p}{(}\PY{l+s+s2}{\PYZdq{}}\PY{l+s+s2}{Mittelwerte von 5 Beobachtungen}\PY{l+s+s2}{\PYZdq{}}\PY{p}{)}
         
         \PY{c+c1}{\PYZsh{} 4. QQ\PYZhy{}Normalplot erstellen}
         \PY{n}{plt}\PY{o}{.}\PY{n}{subplot}\PY{p}{(}\PY{l+m+mi}{2}\PY{p}{,}\PY{l+m+mi}{2}\PY{p}{,}\PY{l+m+mi}{2}\PY{p}{)}
         \PY{n}{st}\PY{o}{.}\PY{n}{probplot}\PY{p}{(}\PY{n}{sim\PYZus{}mean}\PY{p}{,}\PY{n}{plot}\PY{o}{=}\PY{n}{plt}\PY{p}{)}
         \PY{n}{plt}\PY{o}{.}\PY{n}{title}\PY{p}{(}\PY{l+s+s2}{\PYZdq{}}\PY{l+s+s2}{Normal Q\PYZhy{}Q Plot}\PY{l+s+s2}{\PYZdq{}}\PY{p}{)}
         
         \PY{c+c1}{\PYZsh{} Das Ganze nochmals mit 50 Beobachtungen}
         \PY{n}{n} \PY{o}{=} \PY{l+m+mi}{50}
         \PY{n}{sim} \PY{o}{=} \PY{n}{Series}\PY{p}{(}\PY{n}{np}\PY{o}{.}\PY{n}{random}\PY{o}{.}\PY{n}{choice}\PY{p}{(}\PY{n}{werte}\PY{p}{,} \PY{n}{size}\PY{o}{=}\PY{n}{n}\PY{o}{*}\PY{l+m+mi}{1000}\PY{p}{,} \PY{n}{replace}\PY{o}{=}\PY{k+kc}{True}\PY{p}{)}\PY{p}{)}
         \PY{n}{sim} \PY{o}{=} \PY{n}{DataFrame}\PY{p}{(}\PY{n}{np}\PY{o}{.}\PY{n}{reshape}\PY{p}{(}\PY{n}{sim}\PY{p}{,} \PY{p}{(}\PY{n}{n}\PY{p}{,} \PY{l+m+mi}{1000}\PY{p}{)}\PY{p}{)}\PY{p}{)}
         \PY{n}{sim\PYZus{}mean} \PY{o}{=} \PY{n}{sim}\PY{o}{.}\PY{n}{mean}\PY{p}{(}\PY{p}{)}
         
         \PY{n}{plt}\PY{o}{.}\PY{n}{subplot}\PY{p}{(}\PY{l+m+mi}{2}\PY{p}{,}\PY{l+m+mi}{2}\PY{p}{,}\PY{l+m+mi}{3}\PY{p}{)}
         \PY{n}{sim\PYZus{}mean}\PY{o}{.}\PY{n}{hist}\PY{p}{(}\PY{n}{edgecolor}\PY{o}{=}\PY{l+s+s2}{\PYZdq{}}\PY{l+s+s2}{white}\PY{l+s+s2}{\PYZdq{}}\PY{p}{)}
         \PY{n}{plt}\PY{o}{.}\PY{n}{title}\PY{p}{(}\PY{l+s+s2}{\PYZdq{}}\PY{l+s+s2}{Mittelwerte von 50 Beobachtungen}\PY{l+s+s2}{\PYZdq{}}\PY{p}{)}
         
         \PY{n}{plt}\PY{o}{.}\PY{n}{subplot}\PY{p}{(}\PY{l+m+mi}{2}\PY{p}{,}\PY{l+m+mi}{2}\PY{p}{,}\PY{l+m+mi}{4}\PY{p}{)}
         \PY{n}{st}\PY{o}{.}\PY{n}{probplot}\PY{p}{(}\PY{n}{sim\PYZus{}mean}\PY{p}{,}\PY{n}{plot}\PY{o}{=}\PY{n}{plt}\PY{p}{)}
         \PY{n}{plt}\PY{o}{.}\PY{n}{title}\PY{p}{(}\PY{l+s+s2}{\PYZdq{}}\PY{l+s+s2}{Normal Q\PYZhy{}Q Plot}\PY{l+s+s2}{\PYZdq{}}\PY{p}{)}
         
         \PY{n}{plt}\PY{o}{.}\PY{n}{tight\PYZus{}layout}\PY{p}{(}\PY{p}{)}
         \PY{n}{plt}\PY{o}{.}\PY{n}{show}\PY{p}{(}\PY{p}{)}
\end{Verbatim}


    \begin{Verbatim}[commandchars=\\\{\}]
/Users/Christopher/anaconda3/lib/python3.6/site-packages/numpy/core/fromnumeric.py:52: FutureWarning: reshape is deprecated and will raise in a subsequent release. Please use .values.reshape({\ldots}) instead
  return getattr(obj, method)(*args, **kwds)

    \end{Verbatim}

    \begin{center}
    \adjustimage{max size={0.9\linewidth}{0.9\paperheight}}{output_13_1.png}
    \end{center}
    { \hspace*{\fill} \\}
    
    \begin{center}\rule{0.5\linewidth}{\linethickness}\end{center}

\textbf{c.)} Als nächstes simulieren wir die \textbf{Verteilung von
\(\overline{X}_n\) auch für die Fälle, wo \(\overline{X}_n\) das Mittel
von \(n=10\) resp. \(n=200\) \(X_i\) ist.} Wie oben bereits erläutert
nimmt die Verteilung immer mehr die Streuung der Normalverteilung an je
mehr Beobachtungen durchgeführt werden.

    \begin{Verbatim}[commandchars=\\\{\}]
{\color{incolor}In [{\color{incolor}48}]:} \PY{n}{n} \PY{o}{=} \PY{l+m+mi}{10}
         \PY{n}{sim} \PY{o}{=} \PY{n}{Series}\PY{p}{(}\PY{n}{np}\PY{o}{.}\PY{n}{random}\PY{o}{.}\PY{n}{choice}\PY{p}{(}\PY{n}{werte}\PY{p}{,} \PY{n}{size}\PY{o}{=}\PY{n}{n}\PY{o}{*}\PY{l+m+mi}{1000}\PY{p}{,} \PY{n}{replace}\PY{o}{=}\PY{k+kc}{True}\PY{p}{)}\PY{p}{)}
         \PY{n}{sim} \PY{o}{=} \PY{n}{DataFrame}\PY{p}{(}\PY{n}{np}\PY{o}{.}\PY{n}{reshape}\PY{p}{(}\PY{n}{sim}\PY{p}{,}\PY{p}{(}\PY{n}{n}\PY{p}{,}\PY{l+m+mi}{1000}\PY{p}{)}\PY{p}{)}\PY{p}{)}
         \PY{n}{sim\PYZus{}mean} \PY{o}{=} \PY{n}{sim}\PY{o}{.}\PY{n}{mean}\PY{p}{(}\PY{p}{)}
         
         \PY{n}{plt}\PY{o}{.}\PY{n}{subplot}\PY{p}{(}\PY{l+m+mi}{2}\PY{p}{,}\PY{l+m+mi}{2}\PY{p}{,}\PY{l+m+mi}{1}\PY{p}{)}
         \PY{n}{sim\PYZus{}mean}\PY{o}{.}\PY{n}{hist}\PY{p}{(}\PY{n}{edgecolor}\PY{o}{=}\PY{l+s+s2}{\PYZdq{}}\PY{l+s+s2}{white}\PY{l+s+s2}{\PYZdq{}}\PY{p}{)}
         \PY{n}{plt}\PY{o}{.}\PY{n}{title}\PY{p}{(}\PY{l+s+s2}{\PYZdq{}}\PY{l+s+s2}{Mittelwerte von 10 Beobachtungen}\PY{l+s+s2}{\PYZdq{}}\PY{p}{)}
         
         \PY{n}{plt}\PY{o}{.}\PY{n}{subplot}\PY{p}{(}\PY{l+m+mi}{2}\PY{p}{,}\PY{l+m+mi}{2}\PY{p}{,}\PY{l+m+mi}{2}\PY{p}{)}
         \PY{n}{st}\PY{o}{.}\PY{n}{probplot}\PY{p}{(}\PY{n}{sim\PYZus{}mean}\PY{p}{,}\PY{n}{plot}\PY{o}{=}\PY{n}{plt}\PY{p}{)}
         \PY{n}{plt}\PY{o}{.}\PY{n}{title}\PY{p}{(}\PY{l+s+s2}{\PYZdq{}}\PY{l+s+s2}{Normal Q\PYZhy{}Q Plot}\PY{l+s+s2}{\PYZdq{}}\PY{p}{)}
         
         \PY{n}{n} \PY{o}{=} \PY{l+m+mi}{200}
         \PY{n}{sim} \PY{o}{=} \PY{n}{Series}\PY{p}{(}\PY{n}{np}\PY{o}{.}\PY{n}{random}\PY{o}{.}\PY{n}{choice}\PY{p}{(}\PY{n}{werte}\PY{p}{,} \PY{n}{size}\PY{o}{=}\PY{n}{n}\PY{o}{*}\PY{l+m+mi}{1000}\PY{p}{,} \PY{n}{replace}\PY{o}{=}\PY{k+kc}{True}\PY{p}{)}\PY{p}{)}
         \PY{n}{sim} \PY{o}{=} \PY{n}{DataFrame}\PY{p}{(}\PY{n}{np}\PY{o}{.}\PY{n}{reshape}\PY{p}{(}\PY{n}{sim}\PY{p}{,}\PY{p}{(}\PY{n}{n}\PY{p}{,}\PY{l+m+mi}{1000}\PY{p}{)}\PY{p}{)}\PY{p}{)}
         \PY{n}{sim\PYZus{}mean} \PY{o}{=} \PY{n}{sim}\PY{o}{.}\PY{n}{mean}\PY{p}{(}\PY{p}{)}
         
         \PY{n}{plt}\PY{o}{.}\PY{n}{subplot}\PY{p}{(}\PY{l+m+mi}{2}\PY{p}{,}\PY{l+m+mi}{2}\PY{p}{,}\PY{l+m+mi}{3}\PY{p}{)}
         \PY{n}{sim\PYZus{}mean}\PY{o}{.}\PY{n}{hist}\PY{p}{(}\PY{n}{edgecolor}\PY{o}{=}\PY{l+s+s2}{\PYZdq{}}\PY{l+s+s2}{white}\PY{l+s+s2}{\PYZdq{}}\PY{p}{)}
         \PY{n}{plt}\PY{o}{.}\PY{n}{title}\PY{p}{(}\PY{l+s+s2}{\PYZdq{}}\PY{l+s+s2}{Mittelwerte von 200 Beobachtungen}\PY{l+s+s2}{\PYZdq{}}\PY{p}{)}
         
         \PY{n}{plt}\PY{o}{.}\PY{n}{subplot}\PY{p}{(}\PY{l+m+mi}{2}\PY{p}{,}\PY{l+m+mi}{2}\PY{p}{,}\PY{l+m+mi}{4}\PY{p}{)}
         \PY{n}{st}\PY{o}{.}\PY{n}{probplot}\PY{p}{(}\PY{n}{sim\PYZus{}mean}\PY{p}{,} \PY{n}{plot}\PY{o}{=}\PY{n}{plt}\PY{p}{)}
         \PY{n}{plt}\PY{o}{.}\PY{n}{title}\PY{p}{(}\PY{l+s+s2}{\PYZdq{}}\PY{l+s+s2}{Normal Q\PYZhy{}Q Plot}\PY{l+s+s2}{\PYZdq{}}\PY{p}{)}
         
         \PY{n}{plt}\PY{o}{.}\PY{n}{tight\PYZus{}layout}\PY{p}{(}\PY{p}{)}
         \PY{n}{plt}\PY{o}{.}\PY{n}{show}\PY{p}{(}\PY{p}{)}
\end{Verbatim}


    \begin{Verbatim}[commandchars=\\\{\}]
/Users/Christopher/anaconda3/lib/python3.6/site-packages/numpy/core/fromnumeric.py:52: FutureWarning: reshape is deprecated and will raise in a subsequent release. Please use .values.reshape({\ldots}) instead
  return getattr(obj, method)(*args, **kwds)

    \end{Verbatim}

    \begin{center}
    \adjustimage{max size={0.9\linewidth}{0.9\paperheight}}{output_15_1.png}
    \end{center}
    { \hspace*{\fill} \\}
    
    Die obenstehenden Graphiken zeigen, dass die Form der Verteilung des
Mittelwerts von unabhängigen Zufallsvariablen auch dann der
Normalverteilung immer ähnlicher wird, wenn die Variablen selber
überhaupt nicht normalverteilt sind. An der \(x\)-Achse sieht man auch,
dass die Varianz immer kleiner wird.

    Wir stellen also fest, dass
\(\overline{X}_n=\frac{U_1+U_2+\ldots+ U_n}{n}\) einer Normalverteilung
folgt. Der Mittelwert \(\overline{X}_n\) ergibt sich aus:

\begin{quote}
\(\mathrm{E}[\overline{X}_n] = \frac{1}{n}\sum_{i=1}^{n}\mathrm{E}(U_i) = \mathrm{E}(U_i) = \frac{1}{3}(0+10+11) = 7\)
\end{quote}

Die Standardabweichung von \(\overline{X}_n\) folgt aus:

\begin{quote}
\(\mathrm{Var}[\overline{X}_n] = \frac{1}{n^2}\sum_{i=1}^{n}\mathrm{Var}(U_i) = \frac{\mathrm{Var}(U_i)}{n} = \frac{1}{n}((0-7)^2*\frac{1}{3}+(10-7)^2*\frac{1}{3}+(11-7)^2*\frac{1}{3}) = \frac{24.67}{n}\)
\end{quote}

Somit ist die Standardabweichung von \(\overline{X}_n\), also der
Standardfehler, gegeben durch: \textgreater{}
\(\sigma_{\overline{X}_n} = \sqrt{\frac{24.67}{n}}\)

    \begin{Verbatim}[commandchars=\\\{\}]
{\color{incolor}In [{\color{incolor}52}]:} \PY{k+kn}{import} \PY{n+nn}{numpy} \PY{k}{as} \PY{n+nn}{np}
         \PY{k+kn}{from} \PY{n+nn}{pandas} \PY{k}{import} \PY{n}{Series}\PY{p}{,} \PY{n}{DataFrame}
         
         \PY{n}{werte} \PY{o}{=} \PY{n}{np}\PY{o}{.}\PY{n}{array}\PY{p}{(}\PY{p}{[}\PY{l+m+mi}{0}\PY{p}{,}\PY{l+m+mi}{10}\PY{p}{,}\PY{l+m+mi}{11}\PY{p}{]}\PY{p}{)}
         \PY{n}{n} \PY{o}{=} \PY{l+m+mi}{200}
         \PY{n}{sim} \PY{o}{=} \PY{n}{Series}\PY{p}{(}\PY{n}{np}\PY{o}{.}\PY{n}{random}\PY{o}{.}\PY{n}{choice}\PY{p}{(}\PY{n}{werte}\PY{p}{,} \PY{n}{size}\PY{o}{=}\PY{n}{n}\PY{o}{*}\PY{l+m+mi}{1000}\PY{p}{,} \PY{n}{replace}\PY{o}{=}\PY{k+kc}{True}\PY{p}{)}\PY{p}{)}
         \PY{n}{sim} \PY{o}{=} \PY{n}{DataFrame}\PY{p}{(}\PY{n}{np}\PY{o}{.}\PY{n}{reshape}\PY{p}{(}\PY{n}{sim}\PY{p}{,}\PY{p}{(}\PY{n}{n}\PY{p}{,}\PY{l+m+mi}{1000}\PY{p}{)}\PY{p}{)}\PY{p}{)}
         
         \PY{n}{sim\PYZus{}mean} \PY{o}{=} \PY{n}{sim}\PY{o}{.}\PY{n}{mean}\PY{p}{(}\PY{p}{)}
         \PY{n}{sim\PYZus{}mean}\PY{o}{.}\PY{n}{mean}\PY{p}{(}\PY{p}{)}
         \PY{n}{sim\PYZus{}mean}\PY{o}{.}\PY{n}{std}\PY{p}{(}\PY{p}{)}
\end{Verbatim}


    \begin{Verbatim}[commandchars=\\\{\}]
/Users/Christopher/anaconda3/lib/python3.6/site-packages/numpy/core/fromnumeric.py:52: FutureWarning: reshape is deprecated and will raise in a subsequent release. Please use .values.reshape({\ldots}) instead
  return getattr(obj, method)(*args, **kwds)

    \end{Verbatim}

\begin{Verbatim}[commandchars=\\\{\}]
{\color{outcolor}Out[{\color{outcolor}52}]:} 0.36519431521159124
\end{Verbatim}
            
    Experiment und Berechnung sind also in guter Uebereinstimmung.
\(\overline{X}_n\) folgt also der Verteilung \(\mathcal{N}(7,0.12)\).

    \subsection{Aufgabe 5.3}\label{aufgabe-5.3}

\begin{quote}
Boxplot, Transformation, \(\mu\), \(\sigma^2\)
\end{quote}

\begin{center}\rule{0.5\linewidth}{\linethickness}\end{center}

\textbf{a.) Boxplot für jede Versuchtsbedingung erstellen}. Die grösste
normale Beobachtung ist die grösste Beobachtung, die höchstens 1.5 *
Quantilsdifferenz vom oberen Quantil entfernt ist. Bei der hohen Dosis
ist diese nicht weit vom oberen Quartil entfernt, bei der Medium-Dosis
ein wenig weiter weg und bei der tiefen Dosis ist die grösste normale
Beobachtung sehr weit vom oberen Quartil entfernt. Die hohe Dosis-Gruppe
hat einen Ausreisser nach oben.

\begin{itemize}
\tightlist
\item
  Die hohe Dosis-Gruppe hat eher symmetrisch-verteilte Messwerte
  (normalverteilt)
\item
  Die mittlere und tiefe Dosis-Gruppe hat eher rechtsschiefe Messwerte
\item
  Je kleiner die Dosis, desto grösser ist die Streuung
\end{itemize}

    \begin{Verbatim}[commandchars=\\\{\}]
{\color{incolor}In [{\color{incolor}59}]:} \PY{k+kn}{import} \PY{n+nn}{pandas} \PY{k}{as} \PY{n+nn}{pd}
         
         \PY{c+c1}{\PYZsh{} Daten einlesen}
         \PY{n}{iron} \PY{o}{=} \PY{n}{pd}\PY{o}{.}\PY{n}{read\PYZus{}table}\PY{p}{(}\PY{l+s+s2}{\PYZdq{}}\PY{l+s+s2}{ironF3.dat}\PY{l+s+s2}{\PYZdq{}}\PY{p}{,} \PY{n}{sep}\PY{o}{=}\PY{l+s+s2}{\PYZdq{}}\PY{l+s+s2}{ }\PY{l+s+s2}{\PYZdq{}}\PY{p}{,} \PY{n}{index\PYZus{}col}\PY{o}{=}\PY{k+kc}{False}\PY{p}{)}
         \PY{c+c1}{\PYZsh{} Boxplot erstellen}
         \PY{n}{iron}\PY{o}{.}\PY{n}{plot}\PY{p}{(}\PY{n}{kind}\PY{o}{=}\PY{l+s+s2}{\PYZdq{}}\PY{l+s+s2}{box}\PY{l+s+s2}{\PYZdq{}}\PY{p}{,} \PY{n}{title}\PY{o}{=}\PY{l+s+s2}{\PYZdq{}}\PY{l+s+s2}{Boxplot Methode A}\PY{l+s+s2}{\PYZdq{}}\PY{p}{)}
\end{Verbatim}


\begin{Verbatim}[commandchars=\\\{\}]
{\color{outcolor}Out[{\color{outcolor}59}]:} <matplotlib.axes.\_subplots.AxesSubplot at 0x1a18151fd0>
\end{Verbatim}
            
    \begin{center}
    \adjustimage{max size={0.9\linewidth}{0.9\paperheight}}{output_21_1.png}
    \end{center}
    { \hspace*{\fill} \\}
    
    \begin{Verbatim}[commandchars=\\\{\}]
{\color{incolor}In [{\color{incolor}69}]:} \PY{k+kn}{import} \PY{n+nn}{matplotlib}\PY{n+nn}{.}\PY{n+nn}{pyplot} \PY{k}{as} \PY{n+nn}{plt}
         \PY{k+kn}{import} \PY{n+nn}{numpy} \PY{k}{as} \PY{n+nn}{np}
         \PY{k+kn}{from} \PY{n+nn}{pandas} \PY{k}{import} \PY{n}{DataFrame}
         
         \PY{c+c1}{\PYZsh{} iron Daten beschreiben (optional als Übersicht)}
         \PY{n}{DataFrame}\PY{o}{.}\PY{n}{describe}\PY{p}{(}\PY{n}{iron}\PY{p}{)}
\end{Verbatim}


\begin{Verbatim}[commandchars=\\\{\}]
{\color{outcolor}Out[{\color{outcolor}69}]:}             high     medium       low
         count  18.000000  18.000000  18.00000
         mean    3.698889   8.203889  11.75000
         std     2.030870   5.447386   7.02815
         min     0.710000   2.200000   2.25000
         25\%     2.420000   4.320000   6.10250
         50\%     3.475000   5.965000   9.98000
         75\%     4.472500  11.182500  15.99750
         max     8.240000  18.590000  29.13000
\end{Verbatim}
            
    \begin{center}\rule{0.5\linewidth}{\linethickness}\end{center}

\textbf{b.) Logarithmus-Transformation.}

\begin{itemize}
\tightlist
\item
  Wenn man die Daten logarithmiert, so wird die Varianz "stabilisiert",
\item
  d.h. alle Gruppen zeigen jetzt eine ähnlich grosse Streuung.
\item
  Der Unterschied in der Lage ist immer noch ersichtlich.
\end{itemize}

    \begin{Verbatim}[commandchars=\\\{\}]
{\color{incolor}In [{\color{incolor}61}]:} \PY{k+kn}{import} \PY{n+nn}{matplotlib}\PY{n+nn}{.}\PY{n+nn}{pyplot} \PY{k}{as} \PY{n+nn}{plt}
         
         \PY{c+c1}{\PYZsh{} 1. Subplot erstellen (1 Reihe, 2 Zeilen, 1. Plot)}
         \PY{n}{plt}\PY{o}{.}\PY{n}{subplot}\PY{p}{(}\PY{l+m+mi}{1}\PY{p}{,} \PY{l+m+mi}{2}\PY{p}{,} \PY{l+m+mi}{1}\PY{p}{)}
         
         \PY{c+c1}{\PYZsh{} iron Daten als Boxplot plotten}
         \PY{n}{iron}\PY{o}{.}\PY{n}{plot}\PY{p}{(}\PY{n}{kind}\PY{o}{=}\PY{l+s+s2}{\PYZdq{}}\PY{l+s+s2}{box}\PY{l+s+s2}{\PYZdq{}}\PY{p}{,} \PY{n}{ax}\PY{o}{=}\PY{n}{plt}\PY{o}{.}\PY{n}{gca}\PY{p}{(}\PY{p}{)}\PY{p}{)}
         \PY{n}{plt}\PY{o}{.}\PY{n}{ylabel}\PY{p}{(}\PY{l+s+s2}{\PYZdq{}}\PY{l+s+s2}{iron}\PY{l+s+s2}{\PYZdq{}}\PY{p}{)}
         
         \PY{c+c1}{\PYZsh{} 2. Subplot erstellen}
         \PY{n}{plt}\PY{o}{.}\PY{n}{subplot}\PY{p}{(}\PY{l+m+mi}{1}\PY{p}{,} \PY{l+m+mi}{2}\PY{p}{,} \PY{l+m+mi}{2}\PY{p}{)}
         
         \PY{c+c1}{\PYZsh{} log(iron) Daten als Boxplot plotten}
         \PY{n}{np}\PY{o}{.}\PY{n}{log}\PY{p}{(}\PY{n}{iron}\PY{p}{)}\PY{o}{.}\PY{n}{plot}\PY{p}{(}\PY{n}{kind}\PY{o}{=}\PY{l+s+s2}{\PYZdq{}}\PY{l+s+s2}{box}\PY{l+s+s2}{\PYZdq{}}\PY{p}{,} \PY{n}{ax}\PY{o}{=}\PY{n}{plt}\PY{o}{.}\PY{n}{gca}\PY{p}{(}\PY{p}{)}\PY{p}{)}
         \PY{n}{plt}\PY{o}{.}\PY{n}{ylabel}\PY{p}{(}\PY{l+s+s2}{\PYZdq{}}\PY{l+s+s2}{log(iron)}\PY{l+s+s2}{\PYZdq{}}\PY{p}{)}
         
         \PY{n}{plt}\PY{o}{.}\PY{n}{tight\PYZus{}layout}\PY{p}{(}\PY{p}{)}
         \PY{n}{plt}\PY{o}{.}\PY{n}{show}\PY{p}{(}\PY{p}{)}
\end{Verbatim}


    \begin{center}
    \adjustimage{max size={0.9\linewidth}{0.9\paperheight}}{output_24_0.png}
    \end{center}
    { \hspace*{\fill} \\}
    
    \begin{center}\rule{0.5\linewidth}{\linethickness}\end{center}

\textbf{c.) Normalverteilung vor und nach Logarithmieren.}

    \begin{Verbatim}[commandchars=\\\{\}]
{\color{incolor}In [{\color{incolor}76}]:} \PY{k+kn}{import} \PY{n+nn}{matplotlib}\PY{n+nn}{.}\PY{n+nn}{pyplot} \PY{k}{as} \PY{n+nn}{plt}
         \PY{k+kn}{import} \PY{n+nn}{numpy} \PY{k}{as} \PY{n+nn}{np}
         \PY{k+kn}{import} \PY{n+nn}{pandas} \PY{k}{as} \PY{n+nn}{pd}
         \PY{k+kn}{import} \PY{n+nn}{scipy}\PY{n+nn}{.}\PY{n+nn}{stats} \PY{k}{as} \PY{n+nn}{st}
         
         \PY{c+c1}{\PYZsh{} Plot vor dem Logarithmieren}
         \PY{n}{plt}\PY{o}{.}\PY{n}{subplot}\PY{p}{(}\PY{l+m+mi}{1}\PY{p}{,} \PY{l+m+mi}{2}\PY{p}{,} \PY{l+m+mi}{1}\PY{p}{)}
         \PY{n}{st}\PY{o}{.}\PY{n}{probplot}\PY{p}{(}\PY{n}{iron}\PY{p}{[}\PY{l+s+s2}{\PYZdq{}}\PY{l+s+s2}{medium}\PY{l+s+s2}{\PYZdq{}}\PY{p}{]}\PY{p}{,} \PY{n}{plot}\PY{o}{=}\PY{n}{plt}\PY{p}{)}
         \PY{n}{plt}\PY{o}{.}\PY{n}{title}\PY{p}{(}\PY{l+s+s2}{\PYZdq{}}\PY{l+s+s2}{Eisenwerte (mittel) vor Log.}\PY{l+s+s2}{\PYZdq{}}\PY{p}{)}
         
         \PY{c+c1}{\PYZsh{} Plot nach dem Logarithmieren}
         \PY{n}{plt}\PY{o}{.}\PY{n}{subplot}\PY{p}{(}\PY{l+m+mi}{1}\PY{p}{,} \PY{l+m+mi}{2}\PY{p}{,} \PY{l+m+mi}{2}\PY{p}{)}
         \PY{n}{st}\PY{o}{.}\PY{n}{probplot}\PY{p}{(}\PY{n}{np}\PY{o}{.}\PY{n}{log}\PY{p}{(}\PY{n}{iron}\PY{p}{[}\PY{l+s+s2}{\PYZdq{}}\PY{l+s+s2}{medium}\PY{l+s+s2}{\PYZdq{}}\PY{p}{]}\PY{p}{)}\PY{p}{,} \PY{n}{plot}\PY{o}{=}\PY{n}{plt}\PY{p}{)}
         \PY{n}{plt}\PY{o}{.}\PY{n}{title}\PY{p}{(}\PY{l+s+s2}{\PYZdq{}}\PY{l+s+s2}{Eisenwerte (mittel) nach Log.}\PY{l+s+s2}{\PYZdq{}}\PY{p}{)}
         
         \PY{n}{plt}\PY{o}{.}\PY{n}{tight\PYZus{}layout}\PY{p}{(}\PY{p}{)}
         \PY{n}{plt}\PY{o}{.}\PY{n}{show}\PY{p}{(}\PY{p}{)}
\end{Verbatim}


    \begin{center}
    \adjustimage{max size={0.9\linewidth}{0.9\paperheight}}{output_26_0.png}
    \end{center}
    { \hspace*{\fill} \\}
    
    \begin{center}\rule{0.5\linewidth}{\linethickness}\end{center}

\textbf{d.) Parameter \(\mu\) (Erwartungswert) und \(\sigma^2\)
(empirische Varianz) schätzen. Wie gross ist Wahrscheinlichkeit, dass
50\% Eisen zurückgehalten wird?}

\begin{itemize}
\tightlist
\item
  Erwartungswert durch empirischen Mittelwert der Daten bei mittlerer
  Dosierung berechnen
\item
  Varianz durch \texttt{iron{[}"medium"{]}.var()} berechnen
\item
  Beide Werte werden benötigt, um die Wahrscheinlichkeit zu berechnen
\end{itemize}

Wenn \(X\) den zurückgehaltenen Prozentsatz Eisen bei mittlerer
Dosierung bezeichnet, dann ist

\begin{quote}
\(X \sim \mathcal{N}(\hat{\mu} = 8.20, \hat{\sigma}^2 = 29.7)\)
\end{quote}

Die Wahrscheinlichkeit \(P(X > 10) = 1 - P(X \leq 10)\) ergibt
0.370805119544

    \begin{Verbatim}[commandchars=\\\{\}]
{\color{incolor}In [{\color{incolor}77}]:} \PY{c+c1}{\PYZsh{} Erwartungswert berechnen}
         \PY{n}{iron}\PY{p}{[}\PY{l+s+s2}{\PYZdq{}}\PY{l+s+s2}{medium}\PY{l+s+s2}{\PYZdq{}}\PY{p}{]}\PY{o}{.}\PY{n}{mean}\PY{p}{(}\PY{p}{)}
\end{Verbatim}


\begin{Verbatim}[commandchars=\\\{\}]
{\color{outcolor}Out[{\color{outcolor}77}]:} 8.203888888888889
\end{Verbatim}
            
    \begin{Verbatim}[commandchars=\\\{\}]
{\color{incolor}In [{\color{incolor}78}]:} \PY{c+c1}{\PYZsh{} empirische Varianz berechnen}
         \PY{n}{iron}\PY{p}{[}\PY{l+s+s2}{\PYZdq{}}\PY{l+s+s2}{medium}\PY{l+s+s2}{\PYZdq{}}\PY{p}{]}\PY{o}{.}\PY{n}{var}\PY{p}{(}\PY{p}{)}
\end{Verbatim}


\begin{Verbatim}[commandchars=\\\{\}]
{\color{outcolor}Out[{\color{outcolor}78}]:} 29.67401339869281
\end{Verbatim}
            
    \begin{Verbatim}[commandchars=\\\{\}]
{\color{incolor}In [{\color{incolor}79}]:} \PY{k+kn}{import} \PY{n+nn}{scipy}\PY{n+nn}{.}\PY{n+nn}{stats} \PY{k}{as} \PY{n+nn}{st}
         
         \PY{c+c1}{\PYZsh{} Wahrscheinlichkeit mit der norm.cdf Methode}
         \PY{l+m+mi}{1} \PY{o}{\PYZhy{}} \PY{n}{st}\PY{o}{.}\PY{n}{norm}\PY{o}{.}\PY{n}{cdf}\PY{p}{(}\PY{n}{x}\PY{o}{=}\PY{l+m+mi}{10}\PY{p}{,} \PY{n}{loc}\PY{o}{=}\PY{l+m+mf}{8.204}\PY{p}{,} \PY{n}{scale}\PY{o}{=}\PY{n}{np}\PY{o}{.}\PY{n}{sqrt}\PY{p}{(}\PY{l+m+mf}{29.67}\PY{p}{)}\PY{p}{)}
\end{Verbatim}


\begin{Verbatim}[commandchars=\\\{\}]
{\color{outcolor}Out[{\color{outcolor}79}]:} 0.37080511954367934
\end{Verbatim}
            
    \subsection{Aufgabe 5.4}\label{aufgabe-5.4}

\begin{quote}
Poissonprozess, Momentenmethode, QQ-Plot, empirische/theoretische
Quantile, Exponentialverteilung, Regressionsgerade
\end{quote}

\begin{center}\rule{0.5\linewidth}{\linethickness}\end{center}

\textbf{a.) Momentenmethode verwenden, um Parameter zu schätzen.}

\begin{itemize}
\tightlist
\item
  Angler fängt in 2 Stunden 15 Fische
\item
  Poissonprozess
\item
  Mit welcher Wahrscheinlichkeit dauert es länger als 12 Minuten bis
  nächster Fisch anbeisst
\end{itemize}

\textbf{Theorie} Ablauf der Momentenmethode

\begin{enumerate}
\def\labelenumi{\arabic{enumi}.}
\tightlist
\item
  Daten \(x_1, x_2, ..., x_n\) als Realisierungen von Zufallsvariablen
  \(X_1, X_2, ..., X_n\) auffassen (mit bekannter Verteilung)
\item
  Erwartungswert \(E(X)\) berechnen
\item
  Gleichung nach unbekanntem Parameter auflösen
\item
  Wahrscheinlichkeit berechnen
\end{enumerate}

Von der Poissonverteilung wissen wir folgendes:

\begin{quote}
\(E(X) = \lambda\), \(Var(X) = \lambda\), d.h. \(\mu = \lambda\),
\(\sigma^2 = \lambda\)
\end{quote}

Also haben wir hier:

\begin{quote}
\(P[X \leq x] \approx \phi(\frac{x - \lambda}{\sqrt{\lambda}})\)
\end{quote}

\textbf{Lösung}

\begin{itemize}
\tightlist
\item
  Als Wahrscheinlichkeitsverteilung der Wartezeit T (in Minuten) \(\to\)
  Exponentialverteilung
\item
  D.h. Wahrscheinlichkeitsdichte ist:
  \(f(t) = \lambda \cdot e^{-\lambda t}\) für \(t > 0\)
\item
  Wir verwenden jedoch die kummulative Verteilungsfunktion der
  Exponentialverteilung \(F(t) = 1 - e^{\frac{-t}{8}}\) (1 -
  Stammfunktion)
\item
  Parameter \(\lambda\) ermitteln (Momentenmethode)
\end{itemize}

\begin{enumerate}
\def\labelenumi{\arabic{enumi}.}
\item
  Daten auffassen (sind bereits gegeben, 2 h - 15 Fische)
\item
  Erwartungswert berechnen:

  \begin{quote}
  \(E(T) = \frac{1}{\lambda}\) (Momentenmethode, beobachteten Wert für
  mittlere verstrichene Zeit gleich dem Erwartungswert setzen).
  Erwartungswert
  \(= \frac{120 Minuten}{15 Fische} = \frac{1}{\lambda} = 8\)
  \end{quote}
\item
  Gleichung nach unbekanntem Parameter auflösen:

  \begin{quote}
  \(\frac{1}{\lambda} = 8 \Leftrightarrow \hat{\lambda} = \frac{1}{8}\)
  \end{quote}
\item
  Wahrscheinlichkeit berechnen:

  \begin{quote}
  \(P(T > 12) = 1 - P(T \leq 12) = 1 - F(12) = 1 - (1 - e^{-12/8}) = e^{-1.5} = 0.223\)
  \end{quote}
\end{enumerate}

\begin{center}\rule{0.5\linewidth}{\linethickness}\end{center}

\textbf{b.) Wahrscheinlichkeit, dass genau 2 Fische in 12 Minuten
anbeissen.}

\begin{itemize}
\tightlist
\item
  X ist Anzahl Fische, die in nächsten 12 Minuten anbeissen
\item
  X ist poissonverteilt mit \(\lambda = 1.5\) (in 12 Minuten beissen
  durchschnittlich 1.5 Fische an)
\end{itemize}

\begin{quote}
\(P(X = 12) = e^{-1.5} \cdot \frac{1.5^2}{2!} = 0.251\)
\end{quote}

\textbf{c.) Wartezeiten zwischen Fischfängen. QQ-Plot. Steigung der
Regressionsgeraden.}

Für die Exponentialverteilung haben wir:

\begin{quote}
\(F(t) = \begin{cases} 1 - e^{-\lambda t} \text{ falls } t \geq 0\\ 0 \text{ falls } t < 0 \end{cases}\)
\end{quote}

Das \(\alpha\)-Quantil ist gegeben durch die Beziehung
\(F(q_\alpha) = \alpha\), also (Inverse):

\begin{quote}
\(q_\alpha = F^{-1}(\alpha) = - \frac{1}{\lambda} log(1-\alpha)\)
\end{quote}

    \begin{Verbatim}[commandchars=\\\{\}]
{\color{incolor}In [{\color{incolor}5}]:} \PY{k+kn}{from} \PY{n+nn}{pandas} \PY{k}{import} \PY{n}{Series}
        \PY{k+kn}{import} \PY{n+nn}{scipy}\PY{n+nn}{.}\PY{n+nn}{stats} \PY{k}{as} \PY{n+nn}{st}
        \PY{k+kn}{import} \PY{n+nn}{matplotlib}\PY{n+nn}{.}\PY{n+nn}{pyplot} \PY{k}{as} \PY{n+nn}{plt}
        
        \PY{c+c1}{\PYZsh{} 1. Messwerte eintragen und sortieren}
        \PY{n}{messungen} \PY{o}{=} \PY{n}{Series}\PY{p}{(}\PY{p}{[}\PY{l+m+mf}{16.9}\PY{p}{,} \PY{l+m+mf}{4.2}\PY{p}{,} \PY{l+m+mf}{6.7}\PY{p}{,} \PY{l+m+mf}{8.83}\PY{p}{,} \PY{l+m+mf}{10.7}\PY{p}{,} \PY{l+m+mf}{22.4}\PY{p}{,} \PY{l+m+mf}{1.37}\PY{p}{,} \PY{l+m+mi}{3}\PY{p}{,} \PY{l+m+mf}{4.82}\PY{p}{,} \PY{l+m+mf}{4.53}\PY{p}{,} \PY{l+m+mf}{6.77}\PY{p}{,} \PY{l+m+mf}{4.81}\PY{p}{]}\PY{p}{)}
        
        \PY{n}{x} \PY{o}{=} \PY{n}{st}\PY{o}{.}\PY{n}{probplot}\PY{p}{(}\PY{n}{messungen}\PY{p}{,} \PY{n}{plot}\PY{o}{=}\PY{n}{plt}\PY{p}{)}
        \PY{n}{plt}\PY{o}{.}\PY{n}{show}\PY{p}{(}\PY{p}{)}
\end{Verbatim}


    \begin{center}
    \adjustimage{max size={0.9\linewidth}{0.9\paperheight}}{output_32_0.png}
    \end{center}
    { \hspace*{\fill} \\}
    
    \subsection{Aufgabe 5.5}\label{aufgabe-5.5}

    Stetige Verteilung mit gegebener Dichte.

\textbf{a. + b.) Likelihood- und Log-Likelihood-Funktion bestimmen}.
Integral muss 1 geben, damit es eine gültige Wahrscheinlichkeitskurve
ist.

\begin{quote}
\begin{enumerate}
\def\labelenumi{\arabic{enumi}.}
\tightlist
\item
  Integral berechnen
\item
  Gibt Integral 1, dann ist es eine gültige Wahrscheinlichkeitskurve
\item
  Alle x-Werte auf Funktion anwenden
\item
  \(L(\alpha) = f(x_1 ; \alpha) * f(x_2 ; \alpha) *** f(x_5 ; \alpha) = \frac{\alpha}{x_1^{\alpha + 1}} + ... + = \displaystyle\sum_{i=1}^{5} \frac{\alpha}{x_i^{\alpha + 1}}\)
\item
  \(L(\alpha) = log(L(\alpha)) => \frac{5}{log(12) + log(4) + ... + log(15.4)}\)
\end{enumerate}
\end{quote}

    \textbf{c.) Momentenschätzer für \(\alpha\) bestimmen.}

\begin{quote}
E{[}X{]} = Integral(-\(\infty\), \(\infty\)) von
\(xf(x, \alpha)dx = \overline{x_n}\)
\end{quote}

Die Likelihood-Methode ist eher zu vertrauen, wenns um die Genauigkeit
geht.


    % Add a bibliography block to the postdoc
    
    
    
    \end{document}
